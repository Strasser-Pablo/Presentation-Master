\usepackage{amsmath,amssymb}
\usepackage{amsthm}
\newtheorem{property}{Property}

\newcommand{\vect}[1]{ {\boldsymbol {#1}}}


\title[Navier-Stokes]{Simulation of Jet D'Eau from A to Y:\\ Numerical algorithm to solve incompressible Navier-Stokes equations}


\author{Pablo Strasser}


\institute[Unige] % (optional, but mostly needed)
{Univerity of Geneva}


\date[\today] % (optional, should be abbreviation of conference name)
{\today}

\subject{Navier-Stokes equations}


\AtBeginSubsection[]
{
  \begin{frame}<beamer>{Outline}
    \tableofcontents[currentsection,currentsubsection]
  \end{frame}
}


\begin{document}
\begin{frame}
  \titlepage
\end{frame}

\begin{frame}{Outline}
  \tableofcontents[pausesections]
\end{frame}

\section{Notation and basic property}

\begin{frame}
 
\begin{block}{Nabla}
 \begin{equation*}
  \vect{\nabla}=\begin{pmatrix}
                 \partial_1\\
                 \vdots \\
                 \partial_n
                \end{pmatrix}
 \end{equation*}
\end{block}
\end{frame}


\begin{frame}{Differential operator}

\begin{block}{Gradient}
\begin{equation*}
 \vect{\nabla}p=\sum_{i}\partial_{i}p_{i}
\end{equation*}
\end{block}

\begin{block}{Divergence}
\begin{equation*}
 \vect{\nabla}\cdot \vect{v}=\sum_{i}\partial_{i}\vect{v}_{i}
\end{equation*}
\end{block}

\begin{block}{Rotational}
\begin{equation*}
 \vect{\nabla} \times \vect{v}=\begin{pmatrix}
                                \partial_{2}\vect{v}_{3}-\partial_{3}\vect{v}_{2}\\
                                \partial_{3}\vect{v}_{1}-\partial_{1}\vect{v}_{3}\\
                                \partial_{1}\vect{v}_{2}-\partial_{2}\vect{v}_{1}\\
                               \end{pmatrix}
 \end{equation*}
\end{block}


\begin{block}{Convection}
\begin{equation*}
 \left(\vect{v}\cdot \vect{\nabla}\right)\vect{v}=\sum_{i} \vect{v}_{i}\partial_{i} \vect{v}
 \end{equation*}
\end{block}


\end{frame}

\begin{frame}
 \begin{property}{Divergence of a rotational}
  \begin{equation*}
  \nabla \cdot \nabla \times \vect{v}=0
  \end{equation*}
 \end{property}

 \begin{property}{Rotational of a gradient}
  \begin{equation*}
  \nabla \times \nabla p=0
  \end{equation*}
 \end{property}
\end{frame}

\begin{frame}
 
 \begin{property}
 For every vector $\vect{v}$ we can project to a divergence free space without changing the rotational by:
  \begin{align*}
  \vect{v}_{new}&=v-\vect{\nabla} p\\
  \Delta p&=\vect{\nabla} \cdot \vect{v}
  \end{align*}
  \begin{proof}
   \begin{equation*}
   \nabla\cdot \vect{v}_{new}=\vect{\nabla} \cdot \vect{v}-\vect{\nabla} p=\vect{\nabla} \cdot \vect{v}-\vect{\nabla} \cdot \vect{v}=0
   \end{equation*}
   
   \begin{equation*}
    \vect{\nabla} \times \vect{v}_{new}=\vect{\nabla} \times \vect{v}-\vect{\nabla}\times \vect{\nabla}p=\vect{\nabla} \times \vect{v}
   \end{equation*}
  \end{proof}

 \end{property}
 
 \begin{definition}
  \begin{equation*}
   P(\vect{v})=-\vect{\nabla}\Delta^{-1}\vect{\nabla}\cdot \vect{v}
  \end{equation*}

 \end{definition}


\end{frame}

\section{Analytical result}
\begin{frame}
 \begin{block}{Navier-Stokes equations}
\begin{align*}
\vect{\nabla} \cdot \vect{v}(\vect{x} ,t)&=0\\
\partial_t \vect{v}(\vect{x} ,t)&=f(\vect{v}(\vect{x},t))-\vect{\nabla}p
\intertext{Where}
f(\vect{v}(\vect{x},t))&=-(\vect{v}\cdot\vect{\nabla} ) \vect{v}+\frac{\vect{F}}{\rho}+\nu \Delta \vect{v}
\end{align*}
 \end{block}
 
 \begin{block}{Reformulation}
 \begin{align*}
  \partial_t \left(\vect{\nabla}\cdot \vect{v}\right)&=0=\vect{\nabla}\cdot f(\vect{v})-\Delta p\\
  \Delta p&=\vect{\nabla}\cdot f(\vect{v})\\
  \partial_t \vect{v}&=(1+P)f(\vect{v})
  \end{align*}
 \end{block}


\end{frame}

\begin{frame}
 \begin{block}{Characteristic definition}
\begin{align*}
 \partial_t \vect{\xi}_{\lambda}(t)&=\vect{v}(\vect{\xi}_{\lambda}(t),t)\\
 \vect{\xi}_{\lambda}(t_0)&=\vect{\xi}^{0}_{\lambda}
\end{align*}
\end{block}

\begin{block}{Lagrangian speed}
\begin{equation*}
 \vect{u}_{\lambda}(t)=\vect{v}(\vect{\xi}_{\lambda}(t),t)
\end{equation*}
\end{block}

\begin{block}{Material derivatif}
\begin{align*}
\frac{d \vect{u}_{\lambda}(t)}{d t}&=\frac{d \vect{v}(\xi_{\lambda},t)}{d t}=\partial_t \vect{v}+\left(\frac{\partial \vect{\xi}_{\lambda}(t)}{\partial t}\cdot\vect{\nabla}\right)\vect{v}\\
\frac{d \vect{u}_{\lambda}(t)}{d t}&=\partial_t \vect{v}+\left(\vect{v} \cdot\vect{\nabla}\right)\vect{v}
\end{align*}
 \end{block}

\end{frame}
\begin{frame}
 \begin{block}{Lagrangian Navier-Stokes}
 \begin{align*}
\vect{\nabla} \cdot \vect{u}_{\lambda}(t)&=0\\
\frac{d \vect{u}_{\lambda}(t)}{d t}&=-\vect{\nabla} p(\vect{\xi}_{\lambda}(t),t)+\frac{\vect{F}(\vect{\xi}_{\lambda}(t),t)}{\rho(\vect{x},t)}+\nu \Delta \vect{u}_{\lambda}(t)
 \end{align*}
 \end{block}

\end{frame}


\begin{frame}
 \begin{block}{Original}
 \begin{align*}
  \vect{\nabla} \cdot \vect{v}(\vect{x} ,t)&=0\\
\partial_t \vect{v}(\vect{x} ,t)&=f(\vect{v}(\vect{x},t))-\vect{\nabla}p
\end{align*}
 \end{block}
 \begin{block}{Projection}
 \begin{align*}
    \partial_t \left(\vect{\nabla}\cdot \vect{v}\right)&=0=\vect{\nabla}\cdot f(\vect{v})-\Delta p\\
  \Delta p&=\vect{\nabla}\cdot f(\vect{v})\\
  \partial_t \vect{v}&=(1+P)f(\vect{v})
  \end{align*}
 \end{block}

 \begin{block}{Lagrangian}
  \begin{align*}
\vect{\nabla} \cdot \vect{u}_{\lambda}(t)&=0\\
\frac{d \vect{u}_{\lambda}(t)}{d t}&=-\vect{\nabla} p(\vect{\xi}_{\lambda}(t),t)+\frac{\vect{F}(\vect{\xi}_{\lambda}(t),t)}{\rho(\vect{x},t)}+\nu \Delta \vect{u}_{\lambda}(t)
 \end{align*}
 \end{block}


\end{frame}



\end{document}


